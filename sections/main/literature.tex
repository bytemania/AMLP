\clearpage
\section{Literature Review}

Stock market prediction using deep learning and machine learning techniques has been 
a widely researched area in financial forecasting. This section reviews key 
literature that explores the use of LSTM, hybrid models, and technical indicators 
in predicting stock market trends.

\subsection{Traditional Approaches to Stock Market Prediction}

Early methods of stock market prediction relied heavily on statistical models such as 
\acrfull{arima} and \acrfull{garch}. These models were useful for linear trend
analysis but struggled with capturing the non-linear and highly volatile nature of
financial time-series data~\parencite{guo2024LSTMStock}. Similarly, fundamental and
technical analyses have been widely used by traders, relying on financial ratios, 
market trends, and macroeconomic 
factors~\parencite{balasubramanian2023SystematicSurvey}. However, with the increasing
complexity of financial markets, these traditional models have become less effective
in predicting short-term stock movements.

\subsection{Machine Learning-Based Models for Stock Prediction}

With the advent of machine learning, various algorithms have been employed for stock price
forecasting, including \acrfullpl{svm}, \acrfullpl{rf}, and \acrfullpl{gbm} such as 
\emph{XGBoost}~\parencite{nabipour2020DeepLearning}. These models have demonstrated 
improved performance over 
traditional methods by learning complex patterns in stock price data. However, they often 
fail to capture long-term dependencies in time-series data, making them less effective in 
sequential forecasting tasks.

\citetitle{parmar2018stock}~\parencite{parmar2018stock} explored machine learning 
techniques for stock market 
prediction, highlighting the effectiveness of ensemble learning methods in combining 
multiple models for improved accuracy. While these methods enhance prediction performance,
they still rely on feature engineering and lack the ability to extract deep temporal 
patterns from stock price movements.

\subsection{Deep Learning for Stock Market Prediction}

Deep learning techniques, particularly \acrshortpl{rnn} and \acrshortpl{lstm}, have shown
significant improvements in time-series forecasting by retaining sequential dependencies in
data~\parencite{chang2024StockPrediction}. \acrshortpl{lstm} are designed to overcome
the vanishing 
gradient\footnote{The vanishing gradient problem occurs in deep neural networks when
gradients become too small during backpropagation, leading to ineffective weight updates 
and slow or stalled learning. This issue is especially prevalent in \acrshortpl{rnn}, 
where long-range dependencies are difficult to capture. \acrshortpl{lstm} address this by
incorporating gating mechanisms that regulate the flow of information, mitigating the
problem.} problem in \acrshortpl{rnn}, making them ideal for modeling long-term 
dependencies in stock price trends.

A comparative study by 
\citeauthor{nabipour2020DeepLearning}~\parencite{nabipour2020DeepLearning} showed 
that \acrshortpl{lstm} outperformed traditional \acrfull{ml} models such as \acrfullpl{dt}, and 
\acrshortpl{rf} in predicting stock price movements. 

\subsection{Hybrid Deep Learning Models: LSTM-GRU and Advanced Techniques}

Recent studies have explored hybrid models, integrating \acrshort{lstm} with 
\acrfullpl{bigru}\footnote{LSTM-BiGRU is a hybrid model that combines \acrshort{lstm} 
networks and \acrfullpl{bigru}. \acrshortpl{lstm} excel at capturing long-term dependencies in sequential
data, while \acrshortpl{bigru} process information in both forward and backward directions, enhancing
context awareness. This combination is particularly useful in time-series forecasting, improving
predictive accuracy and robustness in stock market prediction.} for improved predictive performance.
\acrshortpl{gru} simplify the \acrshort{lstm} structure while maintaining its memory retention
capabilities, leading to faster training times and better generalization~\parencite{guo2024LSTMStock}.
Other studies have investigated alternative combinations of deep learning and traditional models,
integrating techniques like \acrfull{xgboost} to enhance forecasting
accuracy~\parencite{agrawal2022StockPrediction}. Furthermore, the use of 
Facebook Prophet\footnote{Facebook Prophet is an open-source time-series forecasting tool developed by
Meta. It is designed to handle trends, seasonality, and holiday effects using an additive model. 
Prophet is particularly effective for stock market prediction as it can manage missing data, outliers,
and irregular time-series intervals, making it robust for financial forecasting and trend analysis.} and 
statistical techniques like \acrfull{arima}\footnote{\acrshort{arima} is a time-series forecasting model
that combines AutoRegression, Integration to remove trends, and Moving Average to smooth fluctuations.
It is commonly used for financial forecasting but struggles with highly volatile or non-linear
patterns.} in conjunction with deep learning models has been explored to balance short-term and 
long-term prediction needs~\parencite{phuoc2024StockPrediction}.

\subsection{Incorporating Technical Indicators in Stock Prediction Models}

Technical indicators play a crucial role in stock price forecasting by capturing momentum, volatility, 
and trend reversals. Several studies have incorporated technical indicators such as 
\acrshort{macd}, \acrshort{rsi}, \acrshort{bb}, and \acrshort{sma} into \acrshort{lstm}-based 
prediction models~\parencite{guo2024LSTMStock}.

\citetitle{phuoc2024StockPrediction}~\parencite{phuoc2024StockPrediction} applied \acrshort{macd}, 
\acrshort{rsi}, and \acrshort{sma} in an \acrshort{lstm} model and achieved a 93\% accuracy rate in
forecasting stock trends in the Vietnamese market.   

\citetitle{chang2024StockPrediction}~\parencite{chang2024StockPrediction} extended this approach by 
incorporating ensemble techniques, combining deep learning with statistical models like 
\acrshort{arima} and Facebook Prophet. This hybrid methodology improved prediction stability across
different market conditions, demonstrating the importance of integrating multiple forecasting 
techniques.

\subsection{Challenges}

Despite advancements in deep learning models for stock prediction, several challenges remain. One major
issue is \emph{overfitting}, where models perform well on historical data but fail to generalize 
to unseen market conditions. To mitigate this, studies have employed dropout techniques, 
\emph{L2 regularization}, and ensemble learning~\parencite{shaban2024SMPDL}.

Another challenge is the selection of optimal technical indicators, as not all indicators contribute 
equally to model performance. Future research should focus on automated feature selection techniques
and reinforcement learning-based approaches to dynamically adapt feature selection based on
real-time market conditions~\parencite{balasubramanian2023SystematicSurvey}.

Finally, \acrfull{hft} applications pose additional challenges due to the rapid nature
of stock price fluctuations. Studies such as \citeauthor{guo2024LSTMStock}~\parencite{guo2024LSTMStock}
emphasize the need for specialized deep learning models that can handle millisecond-level trading 
data, further pushing the boundaries of financial \acrfull{ai} research.

\subsection{Key Contributions}

Table~\ref{tab:keycontrib} summarizes key contributions from studies on stock market prediction using deep learning. It highlights advancements in \acrshort{lstm}, \acrshort{gru}, and hybrid models, with notable improvements in forecasting accuracy through the integration of technical indicators and ensemble techniques.

\begin{table}[H]
\centering
\caption{Key Contribution by Study}
\label{tab:keycontrib}
\begin{tabular}{rp{4cm}p{4cm}} \hline
     \textbf{Study} & \textbf{Key Contribution} & \textbf{Demerit} \\ \hline\hline
     \parencite{guo2024LSTMStock} & \acrshort{lstm} based prediction using \acrshort{hft}, 
     incorporating technical indicators for enhanced accuracy.  & Limited generalizability to 
     non-\acrshort{hft} environments. \\
     \parencite{chang2024StockPrediction} & Comparison of \acrshort{lstm} and \acrshort{gru} for 
     stock prediction, demonstrating \acrshortpl{gru} efficiency in training time and performance. & 
     Computationally expensive and requires extensive tuning. \\
     \parencite{nabipour2020DeepLearning} & Demonstrated \acrshortpl{lstm} superiority over 
     traditional \acrshort{ml} models for stock prediction. & Does not account for external economic factors. \\ 
     \parencite{shaban2024SMPDL} & Proposed a novel \acrshort{lstm}-\acrshort{bigru} model, 
     achieving high forecasting accuracy. & Higher model complexity leads to increased training time. \\
     \parencite{phuoc2024StockPrediction} & Applied \acrshort{lstm} with \acrshort{macd}, 
     \acrshort{rsi}, and \acrshort{sma}, achieving 93\% accuracy in Vietnamese stock market 
     predictions. & Results may not generalize well to global markets. \\
     \parencite{agrawal2022StockPrediction} & Used \acrshort{xgboost} and deep learning models to
     improve stock forecasting accuracy. & May overfit to historical data, reducing real-world applicability. \\
     \parencite{parmar2018stock} & Explored machine learning techniques for stock market prediction, 
     highlighting the effectiveness of ensemble learning methods. & Relies on traditional ML models without
     deep learning comparison. \\
     \parencite{balasubramanian2023SystematicSurvey} & Conducted a systematic literature review on 
     \acrshort{ai}-driven stock market prediction techniques. & Does not perform original experiments but 
     relies on findings from other studies. \\ 
     \hline
\end{tabular}
\end{table}

\subsection{Less relevant papers}

The use of volume-based indicators has gained increasing attention in recent research 
on technical analysis and stock prediction. \parencite{tsang2009OBV} provide one of the
few empirical investigations into the profitability of the \acrfull{obv} trading rule.
Their results demonstrate that OBV-based strategies can outperform the traditional
buy-and-hold approach in several Asian markets, especially in Greater China, 
highlighting OBV's potential in capturing investor sentiment through volume dynamics.

Similarly,\parencite{marek2020mfi} examine the \acrfull{mfi} and show that, when 
optimized, MFI-based trading strategies can consistently outperform the standard 
buy-and-hold approach across a range of major S\&P 500 stocks. Their findings 
reinforce the idea that volume-informed oscillators like MFI can significantly
improve the predictive capability of stock models, especially when the parameters 
are tailored to specific equities.

In addition to model architecture and feature engineering, effective hyperparameter tuning plays a critical role in 
enhancing deep learning performance. Traditional grid search methods often suffer from inefficiency, especially when 
dealing with high-dimensional spaces where some parameters have significantly greater impact than others. Bergstra and 
Bengio~\parencite{bergstra2012random} proposed \emph{random search} as a more efficient alternative, showing that randomly 
sampling configurations often yields better results in fewer trials. Their findings are particularly relevant for deep 
neural networks, where hyperparameters such as learning rate, batch size, and dropout have non-uniform effects on model 
performance. In this project, random search is adopted as the primary strategy for tuning \acrshort{lstm} and 
\acrshort{lstmgru} models, as it balances search space coverage with computational efficiency.
