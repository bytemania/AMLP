\clearpage
\pagenumbering{arabic}
\pagestyle{plain}

\section{Introduction}

Stock market prediction has long been a focal point for investors, economists, and
data scientists due to the potential for high financial returns. Traditional
approaches, such as fundamental and technical analysis, have provided insights into
stock price movements. However, the complex and volatile nature of financial markets
requires more sophisticated models capable of capturing intricate patterns in 
time-series data.

With advancements in deep learning, \acrfull{lstm} networks have emerged as a 
powerful tool for time-series forecasting. \acrshort{lstm} networks are a variant 
of \acrfullpl{rnn} designed to retain long-term dependencies, making them highly
suitable for stock price prediction. By incorporating technical indicators such as
\acrfull{macd}, \acrfull{rsi}, \acrfull{bb}, and \acrfull{sma}, \acrshort{lstm}
can improve forecasting accuracy by leveraging historical price trends and market
momentum.

While standalone \acrshort{lstm} models have proven effective in predicting stock
prices, recent studies suggest that hybrid models, such as \acrshort{lstm} combined
with \acrfull{gru}, can further enhance predictive performance. \acrshortpl{gru}, like
\acrshortpl{lstm}, are designed to capture sequential dependencies but utilize a more
simplified gating mechanism, reducing computational complexity while maintaining
accuracy. By integrating \acrshort{lstm} and \acrshort{gru} architectures, researchers
have demonstrated improvements in both training efficiency and forecasting precision,
making them an attractive option for financial time-series modeling.

This research explores the application of \acrshort{lstm}--based models, both
standalone and in combination with \acrshort{gru} for stock market prediction.
Additionally, it investigates the impact of integrating key technical indicators in
enhancing model accuracy. The goal is to determine the effectiveness of these models
in forecasting stock price movements and identifying optimal configurations for 
market prediction.

\subsection{Project description}

This project focuses on predicting the stock trends of 
\href{https://www.nasdaq.com/market-activity/stocks/wday}{Workday Inc. (WDAY)} using deep learning techniques. 
Given the increasing volatility of financial markets, accurate stock price forecasting is essential for
investors, analysts, and financial institutions. Traditional forecasting methods, such as fundamental and
technical analysis, have been widely used, but they often fail to capture the complex, nonlinear patterns 
present in stock price movements.

The aim is to develop and compare advanced deep learning models, specifically \acrfull{lstm} 
and \acrfull{bigru}, for predicting \acrfull{wday} Prices. By leveraging these models, we aim to extract meaningful
trends, reduce prediction errors, and enhance financial decision-making. The research will assess the feasibility
of integrating deep learning-based predictions into real-time financial applications, such as trading algorithms
or stock monitoring dashboards.

\subsection{Project scope}

This research investigates the use of deep learning models, specifically \acrshort{lstm} and hybrid 
\acrfull{lstmgru} approaches, for \acrshort{wday} stock market prediction by leveraging a variety of technical
indicators. The study will focus on assessing the predictive capabilities of these models across different market
conditions and will explore the influence of various technical indicators on forecasting accuracy.

Building upon existing literature, this study will not only evaluate the standard set of indicators such as 
\acrfull{macd}, \acrfull{rsi}, and \acrfull{bb}. Aditionally other indicators may be used such as: 
\acrfull{ema}, \acrfull{wma}, \acrfull{mom}, \acrfull{obv}, and \acrfull{atr}. By testing a broader set of
indicators, this research seeks to identify which features contribute the most to improving model 
performance~\parencite{parmar2018stock, nabipour2020DeepLearning}.

The project will involve an extensive experimental setup where \acrshort{wday} stock price data and technical indicators 
are collected, preprocessed, and fed into various deep learning models. Performance will be assessed using 
established metrics such as \acrfull{rmse}, \acrfull{mae}, and \acrfull{r2}. The study will also compare the 
stability and generalizability of the models under different market conditions, including periods of high
volatility and economic downturns~\parencite{chang2024StockPrediction, phuoc2024StockPrediction}.

Furthermore, the study will explore the potential benefits of combining multiple technical indicators to form a 
composite feature set, evaluating whether certain indicator combinations yield superior predictive performance. 
This approach will help in identifying optimal configurations for \acrshort{wday} stock price forecasting and provide 
insights into best practices for leveraging deep learning in financial 
analysis~\parencite{balasubramanian2023SystematicSurvey, shaban2024SMPDL}.

By expanding the range of indicators and rigorously testing different model architectures, this research will 
contribute to the development of more robust and reliable stock prediction models. The findings will be 
valuable for investors, financial analysts, and AI researchers seeking to integrate deep learning techniques 
with technical analysis for better decision-making in the \acrshort{wday} stock
market~\parencite{guo2024LSTMStock, agrawal2022StockPrediction}.

\subsection{Report layout}

This report is structured to provide a clear and systematic analysis of stock 
market prediction using machine learning and deep learning techniques. Each 
section builds upon the previous one to ensure a logical progression of ideas 
and findings:

\begin{description}
\item [Section 2 - Literature Review] This section examines existing research 
on stock market prediction, highlighting traditional and modern machine 
learning approaches. It discusses key findings from previous studies and 
identifies gaps that this research aims to address.
\item [Section 3 - Methodology] This section details the research design, 
including data collection methods, feature engineering, model selection, and 
evaluation criteria. It describes the techniques used for stock price
forecasting and sentiment analysis.
\item[Section 4 - Implementation] This section outlines the practical 
implementation of the predictive models, detailing the tools, 
technologies, and configurations used. It provides insight into how the
models were trained and optimized.
\item[Section 5 - Results] The experimental results are presented and analyzed 
in this section. It compares the performance of different models using metrics 
such as \acrshort{rmse} and \acrshort{mae}, and evaluates the impact of 
sentiment analysis on prediction accuracy.
\item[Section 6 - Conclusion] This section synthesizes the findings of the 
study, discusses the implications of the results, and outlines potential 
directions for future research.
\item[Appendix] Additional details, including extended datasets, supplementary 
figures, and code references, are provided in the appendix for completeness.
\end{description}



