\clearpage
\pagenumbering{arabic}
\pagestyle{plain}

\section{Introduction}

Stock market prediction has long been a focal point for investors, economists, and
data scientists due to the potential for high financial returns. Traditional
approaches, such as fundamental and technical analysis, have provided insights into
stock price movements. However, the complex and volatile nature of financial markets
requires more sophisticated models capable of capturing intricate patterns in 
time-series data.

With advancements in deep learning, \acrfull{lstm} networks have emerged as a 
powerful tool for time-series forecasting. \acrshort{lstm} networks are a variant 
of \acrfullpl{rnn} designed to retain long-term dependencies, making them highly
suitable for stock price prediction. By incorporating technical indicators such as
\acrfull{macd}, \acrfull{rsi}, \acrfull{bb}, and \acrfull{sma}, \acrshort{lstm}
can improve forecasting accuracy by leveraging historical price trends and market
momentum.

While standalone \acrshort{lstm} models have proven effective in predicting stock
prices, recent studies suggest that hybrid models, such as \acrshort{lstm} combined
with \acrfull{gru}, can further enhance predictive performance. \acrshortpl{gru}, like
\acrshortpl{lstm}, are designed to capture sequential dependencies but utilize a more
simplified gating mechanism, reducing computational complexity while maintaining
accuracy. By integrating \acrshort{lstm} and \acrshort{gru} architectures, researchers
have demonstrated improvements in both training efficiency and forecasting precision,
making them an attractive option for financial time-series modeling.

This research explores the application of \acrshort{lstm}--based models, both
standalone and in combination with \acrshort{gru} for stock market prediction.
Additionally, it investigates the impact of integrating key technical indicators in
enhancing model accuracy. The goal is to determine the effectiveness of these models
in forecasting stock price movements and identifying optimal configurations for 
market prediction.

\subsection{Project description}

This project focuses on forecasting the next-day stock price of 
\href{https://www.nasdaq.com/market-activity/stocks/wday}{Workday Inc. (WDAY)} using deep learning techniques. 
Given the increasing volatility of financial markets, accurate short-term stock prediction, particularly 
for the \emph{next trading day} is crucial for investors, analysts, and financial institutions. While traditional 
forecasting methods, such as fundamental and technical analysis, have been widely adopted, they often fall short in 
capturing the complex, nonlinear dynamics inherent in stock price movements.

The objective of this research is to develop and compare advanced deep learning models, specifically \acrfull{lstm} and 
\acrfull{bigru}, to predict the \acrfull{wday} closing price for the next trading day. By leveraging these architectures, the
goal is to identify short-term trends, minimize prediction errors, and support more informed financial 
decision-making. Furthermore, the project evaluates the potential of integrating deep learning-driven forecasts into
real-time applications, such as algorithmic trading systems or stock monitoring tools.

\subsection{Project scope}

This research explores the application of deep learning models—specifically 
\acrfull{lstm} and a hybrid \acrfull{lstmbigru} architecture for next-day stock price forecasting
of \acrfull{wday}. The models are trained using a combination of raw market data, 
including Open, High, Low, Close, and Volume (OHLCV) prices, as well as a diverse set of technical 
indicators commonly used in financial analysis.

The primary goal is to evaluate and compare the predictive capabilities of the 
\acrshort{lstm} and \acrshort{lstmbigru} models under varying market conditions. In doing so, the 
study will examine how different technical features influence forecasting performance. Alongside 
standard indicators such as the \acrfull{macd}, \acrfull{rsi}, and \acrfull{bb} will be evaluated.

Building on insights from previous literature~\parencite{parmar2018stock, nabipour2020DeepLearning}, 
the project will implement a structured experimental pipeline. This includes the collection and 
preprocessing of historical stock and indicator data, followed by model training and evaluation using
metrics such as \acrfull{rmse}, \acrfull{mae}, and the \acrfull{r2} as recommended by the reference 
literature~\parencite{chang2024StockPrediction, phuoc2024StockPrediction}.

By evaluating model architectures, input feature sets, and prediction outcomes, this study contributes
toward more effective and interpretable applications of deep learning in financial forecasting. The 
findings offer practical insights for investors, financial analysts, and AI researchers interested in 
leveraging sequential modeling techniques for stock price 
prediction~\parencite{balasubramanian2023SystematicSurvey, shaban2024SMPDL, guo2024LSTMStock, 
agrawal2022StockPrediction}.

\subsection{Report layout}

This report is structured to provide a clear and systematic analysis of stock 
market prediction using machine learning and deep learning techniques. Each 
section builds upon the previous one to ensure a logical progression of ideas 
and findings:

\begin{description}
\item [Section 2 - Literature Review] This section examines existing research 
on stock market prediction, highlighting traditional and modern machine 
learning approaches. It discusses key findings from previous studies and 
identifies gaps that this research aims to address.
\item [Section 3 - Methodology] This section details the research design, 
including data collection methods, feature engineering, model selection, and 
evaluation criteria. It describes the techniques used for stock price
forecasting and sentiment analysis.
\item[Section 4 - Implementation] This section outlines the practical 
implementation of the predictive models, detailing the tools, 
technologies, and configurations used. It provides insight into how the
models were trained and optimized.
\item[Section 5 - Results] The experimental results are presented and analyzed 
in this section. It compares the performance of different models using metrics 
such as \acrshort{rmse} and \acrshort{mae}, and evaluates the impact of 
sentiment analysis on prediction accuracy.
\item[Section 6 - Conclusion] This section synthesizes the findings of the 
study, discusses the implications of the results, and outlines potential 
directions for future research.
\item[Appendix] Additional details, including extended datasets, supplementary 
figures, and code references, are provided in the appendix for completeness.
\end{description}



