\clearpage
\subsection{Exploratory Data Analysis}
\label{app:eda}

For a stock market prediction dataset, conducting Exploratory Data Analysis (EDA) is crucial
to understand trends, seasonality, correlations, stationarity, and feature relationships. 
This process aids in preparing the data for deep learning models like \acrshort{lstm}
or \acrshort{gru} networks. Such analyses help in capturing the temporal dependencies and
patterns inherent in financial time-series data, thereby enhancing the predictive
performance of these models. 


\subsubsection{General Overview}
The dataset consists of \textbf{2,763 records} spanning from \textbf{2014 to 2024} and 
contains stock prices, volume, and various technical indicators.
The \textbf{closing price (close)} ranges from \textbf{\$48.90} to \textbf{\$307.21}, with
a mean of \textbf{\$159.24}.

The \textbf{trading volume (volume)} varies significantly, with a mean
of \textbf{1.97M shares} but reaching a maximum of \textbf{46.87M shares}, indicating 
some high-activity trading days.

\subsubsection{Stock Price Statistics}

\begin{table}[H]
    \centering
    \caption{Stock Price Statistics}
    \label{tab:stock_price_stats}
    \begin{tabular}{lcccc}
        \hline
        & \textbf{Open (\$)} & \textbf{High (\$)} & \textbf{Low (\$)} & \textbf{Close (\$)} \\
        \hline\hline
        \textbf{Mean} & 159.17 & 161.52 & 156.83 & 159.24 \\
        \textbf{Std Dev} & 67.89 & 68.72 & 67.07 & 67.88 \\
        \textbf{Min} & 48.36 & 50.63 & 47.32 & 48.90 \\
        \textbf{Max} & 309.10 & 311.28 & 306.60 & 307.21 \\
        \hline
    \end{tabular}
\end{table}

\paragraph{Interpretation:}
Stock prices \textbf{fluctuate widely}, with a standard deviation of \textbf{67.88}, indicating substantial volatility over time.
The \textbf{minimum price was \$48.90}, while the \textbf{maximum price reached \$307.21}, showing significant price growth.

\subsubsection{Moving Averages \& Trend Indicators}

\begin{table}[H]
    \centering
    \caption{Moving Averages \& Trend Indicators}
    \label{tab:moving_averages}
    \begin{tabular}{lccc}
        \hline
        & \textbf{SMA (\$)} & \textbf{EMA (\$)} & \textbf{WMA (\$)} \\
        \hline\hline
        \textbf{Mean} & 158.97 & 158.97 & 159.06 \\
        \textbf{Std Dev} & 67.64 & 67.57 & 67.69 \\
        \textbf{Min} & 51.22 & 53.16 & 51.04 \\
        \textbf{Max} & 303.04 & 303.22 & 304.20 \\
        \hline
    \end{tabular}
\end{table}

\paragraph{Interpretation:}
The \acrshort{sma}, \acrshort{ema}, \acrshort{wma} have very close values.
Moving Averages are useful for detecting trends but do not capture short-term volatility.

\subsubsection{Momentum \& Volatility Indicators}
\begin{table}[H]
    \centering
    \caption{Momentum \& Volatility Indicators}
    \label{tab:momentum_volatility}
    \begin{tabular}{lcccccc}
        \hline
         & \textbf{MACD} & \textbf{RSI} & \textbf{ADX} & \textbf{MOM} & \textbf{ROC} & \textbf{BB} \\
        \hline\hline
        \textbf{Mean} & 0.47 & 51.95 & 23.03 & 0.60 & 0.63 & 170.36 \\
        \textbf{Std Dev} & 4.08 & 11.85 & 8.98 & 11.64 & 7.07 & 72.06 \\
        \textbf{Min} & -16.14 & 19.47 & 8.74 & -51.43 & -31.12 & 66.57 \\
        \textbf{Max} & 13.82 & 84.81 & 54.47 & 58.64 & 35.24 & 322.45 \\
        \hline
    \end{tabular}
\end{table}

\paragraph{Interpretation:}
\begin{itemize}
    \item \acrshort{macd}: Mean close to zero, suggesting price trends fluctuate evenly
    over time.
    \item \acrshort{rsi}: Ranges from 19.47 to 84.81, with a mean of 51.95, indicating 
    stock price frequently moves between \emph{overbought (>70)} and 
    \emph{oversold (<30)} conditions.
    \item \acrfull{adx}\footnote{ADX measures the strength of a trend but does not 
    indicate its direction. A value above 25 suggests a strong trend, while 
    below 25 indicates a weak or no trend.} Mean of 23.03, suggesting the market is often in
    \emph{non-trending conditions} (values >25 indicate strong trends).
    \item \acrshort{mom} NS \acrfull{roc}\footnote{ROC is a momentum indicator that measures the percentage change in stock price over a specific period. Positive values indicate price growth, while negative values suggest decline.}: Capture short-term movements; some 
    extreme values indicate rapid price changes.
\end{itemize}

\subsubsection{Key Takeaways}

\begin{itemize}
    \item \textbf{Stock prices show high volatility}, which is important for predictive
    models like \acrshort{lstm}. 
    \\Available at: \href{https://www.businessinsider.com/stock-market-outlook-danger-zone-sp500-200-day-moving-average-2025-3}{Business Insider}.
    \item \textbf{Trend indicators} are useful for detecting long-term movements. 
    \\Available at: \href{https://www.marketwatch.com/story/some-market-signals-arent-working-like-they-used-to-heres-the-one-to-watch-23a4e43a}{MarketWatch}.
    \item \textbf{Momentum indicators} highlight price strength \& reversals. 
    \\Available at: \href{https://en.wikipedia.org/wiki/Stochastic_oscillator}{Wikipedia - Stochastic Oscillator}.
    \item \textbf{Volatility indicators} help assess market risk. 
    Available at: \href{https://en.wikipedia.org/wiki/Chaikin_Analytics}{Wikipedia - Chaikin Volatility Indicator}.
\end{itemize}





