\clearpage
\subsection{Exploratory Data Analysis}
\label{app:eda}

For a stock market prediction dataset, conducting Exploratory Data Analysis (EDA) is crucial
to understand trends, seasonality, correlations, stationarity, and feature relationships. 
This process aids in preparing the data for deep learning models like \acrshort{lstm}
or \acrshort{gru} networks. Such analyses help in capturing the temporal dependencies and
patterns inherent in financial time-series data, thereby enhancing the predictive
performance of these models. 


\subsubsection{General Overview}
This dataset spans from 2024-01-01 to 2025-03-28 and includes 2,820 entries. 
Summary statistics show that the average close price has a similar mean of 
161.20. Trading volume averages around 1,974,625 shares, 
peaking near 46.9 million.

\subsubsection{Statistics}

\begin{table}[H]
\centering
\caption{Basic OHLCV Statistics}
\label{tab:ohlcv-stats}
\begin{tabular}{lrrrr}
\hline
\textbf{Measure} & \textbf{Mean} & \textbf{Std Dev} & \textbf{Min} & \textbf{Max}\\
\hline\hline
\textbf{Open (\$)}  & 161.16 & 68.62  & 48.36   & 309.10   \\
\textbf{High (\$)}  & 163.51 & 69.43  & 50.63   & 311.28   \\
\textbf{Low (\$)}   & 158.77 & 67.76  & 47.32   & 306.60   \\
\textbf{Close (\$)} & 161.20 & 68.57  & 48.90   & 307.21   \\
\textbf{Volume}     & 1974625.0 & 1582578.9 & 282700  & 46874100 \\
\hline
\end{tabular}
\end{table}

The Table~\ref{tab:ohlcv-stats} provides a concise snapshot of the stock’s
daily price behavior 
and trading activity across the observed period. The mean values for Open, 
High, Low, and Close hover in the 160–164 dollar range, indicating a 
\emph{relatively consistent price level}. The standard deviations in the high 60s
reveal \emph{moderate volatility within that range}. Notably, the max values 
show the highest points reaching just above \$300, while the volume exhibits a
wide range, from roughly 282,700 to nearly 47 million shares. A wide range in 
trading volume indicates that the number of shares changing hands can fluctuate
considerably. In some periods, the market sees intense participation, where many
buyers and sellers are active (leading to very high trading volume), while at other
times, fewer trades occur (resulting in low volume). This disparity often reflects 
shifting investor sentiment, significant news announcements, or events that cause 
spikes in activity, contrasted with relatively calm periods in which traders are 
less active.

\begin{table}[ht]
\centering
\caption{Summary of Exponential Moving Average (Trend Indicator)}
\label{tab:moving-averages}
\begin{tabular}{lrrrr}
\hline
\textbf{Measure} & \textbf{Mean} & \textbf{Std Dev} & \textbf{Min} & \textbf{Max}\\
\hline\hline
\textbf{EMA} & 160.96 & 68.31 & 53.16 & 303.22 \\
\hline
\end{tabular}
\end{table}

The \acrfull{ema} has a mean of roughly 161, with values spanning from about
53 to over 300. This fairly wide spread suggests that the underlying 
price experienced significant fluctuations over the observation period, causing 
the \acrshort{ema} to range accordingly. The standard deviation of 
roughly 68 further highlights variability in the stock’s short-term price 
action, as the \acrshort{ema} quickly adjusts to shifts in price relative to 
longer-term moving averages.

\begin{table}[H]
\centering
\caption{Summary of Momentum Indicators}
\label{tab:momentum-indicators}
\begin{tabular}{lrrrr}
\hline
\textbf{Indicator} & \textbf{Mean} & \textbf{Std Dev} & \textbf{Min} & \textbf{Max}\\
\hline\hline
\textbf{MACD}         & 0.4298   & 4.0647   & -16.1352 & 13.8299  \\
\textbf{RSI}          & 51.8523  & 11.7803  & 19.4654  & 84.8149  \\
\textbf{Stoch Slow D} & 54.3722  & 28.3187  & 2.1824   & 99.2952  \\
\textbf{CCI}          & 12.0905  & 112.3371 & -453.4696 & 358.4688 \\
\textbf{ADX}          & 22.8327  & 8.9904   & 8.7431   & 54.4725  \\
\textbf{MOM}          & 0.5366   & 11.6639  & -51.4400 & 58.6400  \\
\textbf{ROC}          & 0.6020   & 7.0363   & -31.1173 & 35.2462  \\
\textbf{WILLR}        & -45.6425 & 30.3950  & -99.7559 & -0.0000  \\
\hline
\end{tabular}
\end{table}

These momentum indicators each capture different aspects of the stock’s short-term trends and
potential price reversals. For instance, 
the \acrshort{macd} (mean $\approx$ 0.43, std $\approx$ 4.06) 
oscillates between -16.13 and 13.83, illustrating moderate bullish/bearish shifts over time.
The \acrshort{rsi} centers around 52 (out of a possible 0–100 range), implying an 
overall balance between buying and selling pressure but still allowing for extremes at 
19.47 (oversold) and 84.81 (overbought). Meanwhile, the 
\acrshort{cci} (mean $\approx$ 12, min -453.47, max 358.47) shows a wide 
distribution, revealing strong price movements that oscillate above and below typical 
trend lines. Indicators like \acrshort{adx} (mean $\approx$ 23) suggest moderate trend 
strength, while \acrshort{mom} and \acrshort{roc} both display considerable 
variability, reflecting volatility in price momentum. Overall,
the diversity of values highlights fluctuating market conditions and a dynamic interplay
between bullish and bearish forces.

\begin{table}[H]
\centering
\caption{Summary of Volatility Indicator (Bollinger Bands Upper)}
\label{tab:volatility-indicator}
\begin{tabular}{lrrrr}
\hline
\textbf{Measure} & \textbf{Mean} & \textbf{Std Dev} & \textbf{Min} & \textbf{Max}\\
\hline\hline
\textbf{Bollinger Bands Upper} & 172.4750 & 72.8330 & 66.5703 & 322.4500 \\
\hline
\end{tabular}
\end{table}

The \acrshort{bb} Upper figure averages about 172, with a fairly high standard deviation 
near 73, suggesting the stock’s price experienced notable swings during the observation 
period. Values as low as roughly 66 and as high as over 322 illustrate how market 
volatility occasionally expanded the upper boundary considerably, reflecting significant 
upward price movements at times.

\subsubsection*{Key Takeaways from the EDA}

\begin{description}
    \item[Price Range]
    The average Open, High, Low, and Close prices all cluster around \$160--\$164, but their
    standard deviations (\(\sim\$68\)) and maximums (\(\sim\$300\)) reveal episodes of elevated
    volatility.
    Large fluctuations in trading volume (peaking near 47 million shares) indicate active 
    periods triggered by market events or shifts in investor sentiment.

    \item[Exponential Moving Average (Trend)] 
    The \acrshort{ema} centers around \$161, spanning a broad range (\(\approx\$53\) to
    \(\approx\$303\)).
    Its standard deviation (\(\sim\$68\)) suggests frequent short-term price 
    swings, reflecting responsiveness to sudden changes in price trends.

    \item[Momentum Indicators]
    Indicators such as \acrshort{macd}, \acrshort{rsi}, \acrshort{sso} D, and 
    \acrshort{cci} exhibit wide variation, capturing both bullish surges (positive extremes)
    and bearish drawdowns (negative extremes).
    The \acrshort{rsi} averages near 52, suggesting a mostly balanced market, yet swings
    between \(\approx19\) (oversold) and \(\approx85\) (overbought) highlight sporadic
    extremes in sentiment.
    The \acrshort{cci}’s high standard deviation implies notable oscillations from normal
    trend lines, while the \acrshort{adx} value of \(\approx23\) suggests moderate
    trend strength overall.

    \item[Volatility Insights]
    The Upper Bollinger Band mean (\(\approx\$172\)) and wide range (66--322) confirm
    the stock’s propensity for sharp price fluctuations.
    Periods of expanded \acrshort{bb} typically coincide with strong price 
    movements, underscoring heightened volatility.
\end{description}
