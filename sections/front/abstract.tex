\clearpage
\pagestyle{empty} 

\section*{Abstract}

Stock price prediction is a longstanding challenge in quantitative finance due 
to the complex, noisy, and highly dynamic nature of financial markets. This project
investigates the application of deep learning techniques—specifically Long Short-Term
Memory (LSTM) and a hybrid LSTM-Bidirectional GRU (BiGRU) model—for forecasting the 
next-day closing price of Workday Inc. (WDAY) stock. A 10-year dataset of daily stock
prices and technical indicators was compiled and preprocessed to create input 
sequences that capture temporal dependencies over a 30-day window.

Two neural network architectures were implemented: a stacked LSTM and a 
deeper LSTM-BiGRU hybrid model. Hyperparameter tuning was performed using Keras 
Tuner's random search, optimizing key model parameters such as layer sizes, dropout
rates, learning rates, and sequence lengths. Evaluation was conducted using standard 
regression metrics Root Mean Squared Error (RMSE), Mean Absolute Error (MAE), Mean 
Absolute Percentage Error (MAPE), and the Coefficient of Determination ($R^2$) to 
assess predictive accuracy on unseen test data.

The results show that while both models performed well, the LSTM-BiGRU model 
significantly outperformed the standard LSTM in all evaluation metrics, achieving an 
$R^2$ above 0.99. This supports the hypothesis that deeper, hybrid recurrent 
architectures better capture the non-linear patterns and long-term dependencies 
inherent in financial time series. The findings also suggest that minimal dropout 
regularization, combined with early stopping, is effective in preventing overfitting
in small but well-structured datasets.

Overall, the study demonstrates the potential of deep learning models for stock price
forecasting and provides a strong foundation for future work incorporating sentiment
analysis, higher-frequency data, or transformer-based architectures for more robust 
and context-aware predictions. 
   